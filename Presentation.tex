\documentclass{beamer}
\usepackage[bars]{beamerthemetree}
\usepackage{amsmath}
\usepackage{graphicx}
\usetheme{Warsaw}
\usecolortheme{whale}
\title{Econometric 2}
\author{Yi Huang(Anr:353686)}
\institute{Tilburg University}
\date{\today}
\transdissolve<5>
\begin{document}
	\section{Assignment 3}
	\subsection{Personal Solution}
	\begin{frame}
		\titlepage
	\end{frame}
	\begin{frame}[allowframebreaks]
		\frametitle{Answer A}
		\structure{A.i.}
		
		In this question, $Infl$ is the annualized percentage difference of inflation. The difference between current period logarithmic price index for personal consumption expenditures and last period logarithmic price index, the result times 400 is $Infl$. $Infl$ is measured percentage per quarter, that is, logarithmic price index change one percentage in a quarter, $Infl$ changes 400 percentage points.
		\vspace{11pt}
		
		\structure{A.ii.}
		
		Plot the value of $Infl$ from 1963: Q1 through 2012: Q4 following.
		
		Based on the plot, $Infl$ has a stochastic trend since trend is random and varies over time.
		\pause 
		
		\includegraphics[width = .8\textwidth]{Graph1.png}
	\end{frame}
	\begin{frame}[allowframebreaks]
		\frametitle{Answer B}
			\structure{B.i.}
			
			Compute the first four autocorrelations of $∆Infl$, we get following:
			\vspace{11pt}
			
			\includegraphics[width = 1\textwidth]{Graph2.png}
			\vspace{6pt}
			
			\structure{B.ii.}
			
			The plot looks “choppy” and “jagged”.
			In part (i), we calculated the first autocorrelation of $Infl$, the difference become more usual and changes each period, that is why the plot looks “choppy” and “jagged”.
			\pause
			
			\includegraphics[width = .8\textwidth]{Graph3.png}
			

	\end{frame}
	\begin{frame}[allowframebreaks]
		\frametitle{Answer C}
			\structure{C.i.}
			
			The regression of $∆Infl_t$ on $∆Infl_{t-1}:$
			\pause
			
			\includegraphics[width = .8\textwidth]{Graph4.png}
			
			Since we already known 
			\[ΔInfl_t = β_0 + β_1ΔInfl_{t–1} + u_t, β_0 =-0.0029, β_1 =-0.2531,\] that is 
			\[ΔInfl_t = -0.0029- 0.2531ΔInfl_{t–1}.\] 
			According to constant and coefficient, we could predict the change in inflation next quarter, that is, 
			\[ΔInfl_{t+1} = -0.0029- 0.2531ΔInfl_t.\]
			\vspace{11pt}
			
			\structure{C.ii.}
			
			Estimate an AR(2) model for $∆Infl:$
			\pause
			
			\includegraphics[width = 1.0 \textwidth]{Graph5.png}
			
			Compare with estimate results in part (i) and part (ii), all lag variables are statistic significant and Adjusted R squared is increased from 0.0593 to 0.1205, so the AR(2) model is better than an AR(1) model.
			Since the $∆Infl$ may be influenced by last two periods, adding one more lag variable might increase explanation of the model.
			\vspace{11pt}
			
			\structure{C.iii.}
			
			Estimate an AR(p) model for $p=0,…,8:$
			\pause
			
			\includegraphics[width = .8\textwidth]{Graph6.png}
			
			In general, “smaller is better”. In this case, lag length is 2 when chosen by AIC, lag length is 2 when chosen by BIC.
			\vspace{11pt}
			
			\structure{C.iv.}
			
			We know the AR(2) model is 
			\[ΔInfl_t = -0.011- 0.3537ΔInfl_{t–1} - 0.2439ΔInfl_{t–2}.\]
			So
			\[\begin{aligned}
			ΔInfl_{2013:Q1} ={}& -0.011- 0.3537ΔInfl_{2012:Q4} - 0.2439ΔInfl_{2012:Q3}\Rightarrow\\
			&ΔInfl_{2013:Q1} = -0.011- 0.3537×-0.0595 - 0.2439×-0.6153\Rightarrow ΔInfl_{2013:Q1}=0.1601
			\end{aligned}\]
			\vspace{11pt}
			
			\structure{C.v.}
			
			The AR(2) model is 
			\[Infl_t = 0.4992+ 0.6766Infl_{t–1} + 0.1837Infl_{t–2}.\]
			So 
			\[\begin{aligned}
			Infl_{2013:Q1} ={}& 0.4992+ 0.6766Infl_{2012:Q4} + 0.1837Infl_{2012:Q3}\Rightarrow\\
			&ΔInfl_{2013:Q1} = 0.4992+ 0.6766×1.6127 + 0.1837×1.6722\Rightarrow Infl_{2013:Q1}=1.8975
			\end{aligned}\]
	\end{frame}
	\begin{frame}[allowframebreaks]
		\frametitle{Answer D}
		\structure{D.i.}
		
		$H_0: δ = 0$ vs. $H_1: δ < 0$ with two lags of $ΔInfl$
		
		Estimate $ΔInfl= β_0 + δInfl_{t–1} + γ_1ΔInfl_{t–1} + γ_2ΔInfl_{t–2} +u_t$
		
		\includegraphics[width = .8\textwidth]{Graph7.PNG}

	\end{frame}

\end{document}
